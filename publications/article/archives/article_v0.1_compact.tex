\documentclass[twocolumn,10pt]{article}
\usepackage[utf8]{inputenc}
\usepackage[T1]{fontenc}
\usepackage{lmodern}
\usepackage{amsmath,amssymb}
\usepackage{graphicx}
\usepackage{booktabs}
\usepackage{hyperref}
\usepackage[margin=2cm]{geometry}
\usepackage{natbib}
\usepackage{float}

\newcommand{\LCDM}{$\Lambda$CDM}
\newcommand{\chisq}{$\chi^2$}

\title{\textbf{JANUS Cosmological Model Constraints from Type Ia Supernovae:\\Reproduction and Pantheon+ Extension}}

\author{Patrick Guerin\thanks{Corresponding author: pg@gfo.bzh}\\
\small Independent Researcher, Brittany, France\\
\\
\small \textit{Contributions}: P.G. designed study, performed analyses, wrote manuscript.\\
\small \textit{Funding}: None. \textit{Conflicts}: None declared.\\
\small \textit{Data}: \url{https://github.com/PGPLF/JANUS-S}}

\date{January 4, 2026 (v0.1)}

\begin{document}
\maketitle

\begin{abstract}
We reproduce D'Agostini \& Petit (2018) JANUS model constraints using JLA supernovae (740 SNe Ia), obtaining $q_0 = -0.086 \pm 0.014$ with $\chi^2/\text{dof} = 0.88$, validating the original result ($q_0 = -0.087$). Extension to Pantheon+ (1543 SNe Ia) yields $q_0 = -0.035 \pm 0.014$, revealing significant dataset dependence ($\Delta q_0 = 0.05$). Subsample analysis shows $q_0$ evolution: $-0.26$ at $z<0.1$ to $\sim 0$ at high-$z$. Comparison with \LCDM{} shows comparable fits ($\Delta\chi^2/\text{dof} < 4\%$) with slight \LCDM{} preference ($\Delta$AIC $\approx -25$).
\end{abstract}

\section{Introduction}

The JANUS bimetric model \citep{Petit2014,Petit2018,Petit2024} explains cosmic acceleration through positive/negative mass sector interactions, avoiding dark energy. D'Agostini \& Petit (2018) constrained the deceleration parameter to $q_0 = -0.087 \pm 0.015$ using JLA data \citep{Betoule2014}.

We aim to: (1) reproduce the 2018 analysis, (2) extend to Pantheon+ \citep{Brout2022}, and (3) compare with \LCDM{}.

\section{Methods}

\subsection{JANUS Model}
The luminosity distance is:
\begin{equation}
d_L(z) = \frac{c}{H_0} \left[ z + \frac{z^2(1-q_0)}{1 + q_0 z + \sqrt{1 + 2q_0 z}} \right]
\end{equation}
with distance modulus $\mu = 5\log_{10}(d_L/\text{Mpc}) + 25$.

\subsection{Data}
\textbf{JLA}: 740 SNe Ia, $0.01 < z < 1.30$, using $\mu = m_B - M_B + \alpha x_1 - \beta c$ with $\alpha=0.141$, $\beta=3.101$, $M_B=-19.05$.

\textbf{Pantheon+}: 1543 unique SNe Ia, $0.001 < z < 2.26$, using calibrated MU\_SH0ES distances.

\subsection{Fitting}
Minimize $\chi^2 = \sum_i [(\mu_i^{\text{obs}} - \mu_i^{\text{th}} - \delta)/\sigma_i]^2$ via Nelder-Mead. Bootstrap (100 samples) for uncertainties.

\section{Results}

\subsection{2018 Reproduction}
\begin{table}[H]
\centering
\caption{JLA results comparison}
\begin{tabular}{lcc}
\toprule
& This work & Ref. (2018) \\
\midrule
$q_0$ & $-0.086 \pm 0.014$ & $-0.087 \pm 0.015$ \\
$\chi^2$/dof & 0.883 & 0.89 \\
\bottomrule
\end{tabular}
\end{table}

\subsection{Pantheon+ Extension}
\begin{table}[H]
\centering
\caption{Dataset comparison}
\begin{tabular}{lcc}
\toprule
& JLA & Pantheon+ \\
\midrule
$N$ & 740 & 1543 \\
$q_0$ & $-0.086$ & $-0.035$ \\
$\chi^2$/dof & 0.883 & 0.497 \\
\bottomrule
\end{tabular}
\end{table}

\subsection{Redshift Dependence}
\begin{table}[H]
\centering
\caption{Pantheon+ subsamples}
\begin{tabular}{lccc}
\toprule
Range & $N$ & $q_0$ & $\chi^2$/dof \\
\midrule
$z<0.1$ & 583 & $-0.260$ & 0.58 \\
$z<0.5$ & 1333 & $-0.165$ & 0.50 \\
$z<1.0$ & 1518 & $-0.070$ & 0.49 \\
Full & 1543 & $-0.035$ & 0.50 \\
\bottomrule
\end{tabular}
\end{table}

\subsection{JANUS vs \LCDM{}}
\begin{table}[H]
\centering
\caption{Model comparison}
\begin{tabular}{lcccc}
\toprule
Dataset & JANUS & \LCDM{} & $\Delta$AIC \\
\midrule
JLA & 0.883 & 0.852 & $-24$ \\
Pantheon+ & 0.497 & 0.481 & $-26$ \\
\bottomrule
\end{tabular}
\end{table}

\section{Discussion}

The 2018 reproduction is excellent ($\Delta q_0 = 0.001$). The JLA/Pantheon+ discrepancy ($\Delta q_0 = 0.05$) reflects: (1) different $z$ distributions, (2) calibration differences, (3) possible $q_0(z)$ evolution.

The $q_0$ redshift trend (Fig.~\ref{fig:q0}) suggests single-parameter JANUS may be insufficient for extended $z$ ranges, motivating theoretical extensions.

Both models fit comparably; \LCDM{} is slightly preferred statistically but JANUS remains competitive with one free parameter.

\section{Conclusions}

\begin{enumerate}
\item 2018 reproduction validated: $q_0 = -0.086$, $\chi^2$/dof $= 0.88$
\item Pantheon+ yields different $q_0 = -0.035$
\item Evidence for $q_0(z)$ evolution: $-0.26$ to $\sim 0$
\item JANUS and \LCDM{} comparably fit SNe Ia data
\end{enumerate}

\section*{Acknowledgments}
We thank the JLA and Pantheon+ collaborations for public data access.

\bibliographystyle{apalike}
\bibliography{references}

\clearpage
\onecolumn
\section*{Figures}

\begin{figure}[H]
\centering
\includegraphics[width=0.8\textwidth]{figures/fig1_hubble_jla.pdf}
\caption{JLA Hubble diagram with JANUS ($q_0=-0.086$) and \LCDM{} fits.}
\label{fig:jla}
\end{figure}

\begin{figure}[H]
\centering
\includegraphics[width=0.8\textwidth]{figures/fig2_hubble_pantheon.pdf}
\caption{Pantheon+ Hubble diagram with JANUS ($q_0=-0.035$) and \LCDM{} fits.}
\label{fig:pantheon}
\end{figure}

\begin{figure}[H]
\centering
\includegraphics[width=0.65\textwidth]{figures/fig3_q0_evolution.pdf}
\caption{$q_0(z)$ evolution from Pantheon+ subsamples. Dashed: 2018 value.}
\label{fig:q0}
\end{figure}

\begin{figure}[H]
\centering
\includegraphics[width=0.65\textwidth]{figures/fig4_comparison.pdf}
\caption{$\chi^2$/dof comparison: JANUS vs \LCDM{}.}
\label{fig:comp}
\end{figure}

\end{document}
